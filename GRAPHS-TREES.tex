%version of 01-30-18

\chapter{GRAPH-THEORETIC ENTITIES}

\section{Basic Graph-Theoretic Topics}

Graphs provide one of the richest technical and conceptual frameworks
in the world of computing.  They provide concrete representations of
manifold data structures, hence must be well understood in preparation
for a ``Data Structures and Algorithms'' course.  They embody tangible
abstractions of relationships of all sorts, hence must be well
understood in order to discuss entities as varied as web-search
engines and social networks with precision and rigor.  As with most of
the topics we are discussing, graph-oriented concepts must be taught
``in layers''.  All students must be conversant with the use of graphs
to represent and reason about a variety of complicated relationships,
but the degree of sophistication that an individual student requires
depends both on the abilities of the student and the range of
applications that will appear in the student's program.  The
fundamental concepts in this essay should be understood by all
students---although each can be developed with more texture and nuance
within the context of specific application domains.

Many developments in computing technology over recent decades have
made it imperative that graphs no longer be viewed by students as the
static objects introduced, e.g., as abstractions of data struntures.
Applications ranging from databases to web-search engines to social
networks demand an appreciation of graphs as dynamic objects.  This
change in perspective affects many aspects of the mathematical
prerequisites for any CS/CE program.

\subsection{Fundamental Notions}

The basic components of graphs are {\em nodes/vertices} (one
encounters both terms in the literature) and the {\em edges} that
interconnect them.  When a graph is {\em undirected}, an edge connotes
some sort of sibling-like relationship among nodes of ``equal''
status.  When a graph is {\em directed} (sometimes referred to as a
{\em digraph}), an edge connotes an ``unequal'' relationship such as
parenthood or priority or dependence; edges in directed graphs are
often termed {\em arcs}.  In many situations involving directed
graphs, it is important to recognize The {\em dual} of a digraph $G$
is a digraph obtained by reversing all of $G$'s arcs.  One sometimes
encounters situations when arguments about, or operations on, a
digraph $G$ can be ``translated'' to arguments about, or operations
on, $G$'s dual with only clerical effort.  A {\em subgraph} $G'$ of a
graph $G$ is a graph whose nodes are a subset of $G$'s and whose edges
are a subset of $G$'s that interconnect only nodes of $G'$.  A {\em
  path} in an undirected graph is a sequence of nodes within which
every adjacent pair is connected by an edge.  A path is a {\em cycle}
if all nodes in the sequence are distinct, except for the first and
last, which are identical.  Paths and cycles in directed graphs are
defined similarly, except that every adjacent pair of nodes must be
connected by an arc, and all arcs must ``point in the same
direction.''  

{\small\sf Getting formal}.
%
A {\it directed graph} ({\it digraph}, for short) $\g$ is given by a
set of {\it nodes} $\n_{\fg}$ and a set of {\it arcs} (or {\it
  directed edges}) $\a_{\fg}$.  Each arc has the form $(u \rightarrow
v)$, where $u, v \in \n_{\fg}$; we say that this arc goes {\em from}
$u$ {\em to} $v$.  A {\it path} in $\g$ is a sequence of arcs that
share adjacent endpoints, as in the following path from node $u_1$ to
node $u_n$:
\begin{equation}
\label{e.path}
(u_1 \rightarrow u_2), \ (u_2 \rightarrow u_3), \ \ldots, \ (u_{n-2}
        \rightarrow u_{n-1}), \ (u_{n-1} \rightarrow u_n).
\end{equation}
It is sometimes useful to endow the arcs of a digraph with labels from
an alphabet $\Sigma$.  When so endowed, the path (\ref{e.path}) would
be written
\[
(u_1 \stackrel{\lambda_1}{\rightarrow} u_2), \ 
(u_2 \stackrel{\lambda_2}{\rightarrow} u_3), \ \ldots, \ 
(u_{n-2} \stackrel{\lambda_{n-2}}{\rightarrow} u_{n-1}), \ 
(u_{n-1} \stackrel{\lambda_{n-1}}{\rightarrow} u_n),
\]
where the $\lambda_i$ denote symbols from $\Sigma$.  If $u_1 = u_n$,
then we call the preceding path a {\em cycle}.\index{cycle (in a digraph)}

An {\em undirected graph} is obtained from a directed graph by
removing the directionality of the arcs; the thus-beheaded arcs are
called {\em edges}.  Whereas we say: \\
\hspace*{.35in}the {\em arc} $(u,v)$ goes {\em from} node $u$ {\em to}
node $v$ \\
we say: \\
\hspace*{.35in}the undirected edge $\{u,v\}$ goes {\em between} nodes
$u$ and $v$ \\
or, more simply: \\
\hspace*{.35in}the undirected edge $\{u,v\}${\em connects} nodes $u$
and $v$.  \\
{\em Undirected} graphs are usually the default concept, in the
following sense: {\em When $\g$ is described as a ``graph,'' with no
  qualifier ``directed'' or ``undirected,'' it is understood that $\g$
  is an undirected graph.}


\subsection{Graph evolution and decomposition}

Any course on algorithms will discuss graphs, especially trees, that
evolve over time.  Such structures arise in the study of ``classical''
algorithmic topics such minimum-spanning-tree and branch-and-bound
algorithms, as well as in the study of more modern topics such as
social networks and internetworks.  For the most part, the mathematics
already appearing in this essay suffices for these studies: the students
must assimilate new algorithmic notions, not new mathematics.  That
said, students will be challenged to utilize the mathematical devices
that they have (hopefully) mastered in new, more sophisticated, ways.
Instructors who wish to lead their students to even a casual
understanding of emerging modalities and platforms for computing are
thereby challenged to teach required mathematical preliminaries using
exemplars that include these new modalities and exemplars.

A fundamental variety of relevant notions within the study of graphs
reside in the notions known in various contexts via terms such as {\em
  graph separators} or {\em graph bisection}.  The key idea that
underlies these notions is that certain graph-theoretic structures
interconnect their graphs' constituent nodes more or less densely ---
and the type and level of interconnectivity has important algorithmic
consequences.  In such situations, the student must understand how the
phenomenon/a modeled by the graphs of interest are elucidated by the
way a graph can be broken into subgraphs by excising nodes or edges.
When discussing communication-related structures, for instance, graph
are often used to model the individual pairwise communications that
must occur in order to accomplish the desired overall communication
(such as a broadcast).  There is often a provable tradeoff between the
number of such pairwise communications and the overall time for the
completion of the overall task.  As another example, when discussing
social networks, one can pose questions such as: which node in a
network is best to connect to (when joining the network) in order to
best facilitate one's interactions or influence within the community.
The latter topic leads, e.g., to the study of {\em power-law}
networks, a topic that would not be studied in depth in any early
course; indeed, the structure of these networks is not yet well
understood even in advanced settings.

\subsection{Trees}

Within the world of graphs, trees occupy a place of honor.  In their
undirected form, the are defined by the proerty of containing no
cycles.  They then provide the minimal interconnection structure that
provides a path connecting every pair of nodes.  Of course, trees
appear also in directed form.  Among directed trees, a very important
subclass are those that have a {\em root}, i.e., a node that has a
directed path to every other node of the tree.  In applications,
rooted trees represent hierarchical organizations (as do families when
viewed generationally).

More formally: {\em rooted trees} are a class of {\em acyclic}
digraphs.  Paths in trees that start at the root are often called {\em
  branches}.  The {\em acyclicity} of a tree $\t$ means that for any
branch of $\t$ of the form (\ref{e.path}), we cannot have $u_1 = u_n$,
for this would create a cycle.  Each rooted tree $\t$ has a designated
{\em root node} $u_n \in \n_{\ft}$ that resides at the end of a branch
(\ref{e.path}) that starts at $r_{\ft}$ (so $u_1 = r_{\ft}$) is said
to reside at {\em depth} $n-1$ in $\t$; by convention, $r_{\ft}$ is
said to reside at depth $0$.  $\t$'s root $r_{\ft}$ has some number
(possibly $0$) of arcs that go from $r_{\ft}$ to its {\em children,}
each of which thus resides at depth $1$ in $\t$; in turn, each child
has some number of arcs (possibly $0$) to its children, and so on.
(Think of a family tree.)  For each arc $(u \rightarrow v) \in
A_{\ft}$, we call $u$ a {\it parent} of $v$, and $v$ a {\it child} of
$u$, in $\t$; clearly, the depth of each child is one greater than the
depth of its parent.  Every node of $\t$ except for $r_{\ft}$ has
precisely one parent; $r_{\ft}$ has no parents.  A childless node of a
tree is a {\em leaf}.  The transitive extensions of the parent and
child relations are, respectively, the {\em ancestor} and {\em
  descendant} relations.  The {\em degree} of a node $v$ in a tree is
the number of children that the node has, call it $c_v$.  If every
nonleaf node in a tree has the same degree $c$, then we call $c$ the
{\em degree of the tree}.

\ignore{***********
It is sometimes useful to have a symbolic notation for the ancestor
and descendant relations.  To this end, we write $(u \Rightarrow v)$
to indicate that node $u$ is an ancestor of node $v$, or equivalently,
that node $v$ is a descendant of node $u$.  If we decide for some
reason that we are not interested in really distant descendants of the
root of tree $\t$, then we can {\em truncate} $\t$ at a desired depth
$d$ by removing all nodes whose depths exceed $d$.  We thereby obtain
the {\em depth-$d$ prefix} of $\t$.


Figure \ref{fig.graph-samples} depicts an arc-labeled rooted tree $\t$
whose arc labels come from the alphabet $\{a,b\}$.  $\t$'s arc-induced
relationships are listed in Table~\ref{tab.graph-samples}.
\begin{figure}[htb]
\centerline{\epsfig{figure=graph.sample.eps,height=4truecm}}
\caption{An arc-labeled rooted tree $\t$ whose arc labels come from
  the alphabet $\{a,b\}$.  (Arc labels have no meaning; they are just
  for illustration.)
\label{fig.graph-samples}}
\end{figure}
\begin{table}[htb]
{\small
\begin{center}
\fbox{
\begin{tabular}{c||c|c|c|c}
\multicolumn{5}{c}{The arc-labeled rooted tree $\t$ of
  Figure \ref{fig.graph-samples}} \\
\hline
Node            & Children & Parent & Descendants & Ancestors \\
\hline
\hline
$r_{\ft} = u_0$ 
& $u_1$
& none & 
$u_1, u_2, \ldots, u_k, v_1, v_2, \ldots, v_k,
w_1, w_2, \ldots, w_k$
& none \\
\hline
$u_1$
& $u_2, v_1$
& $u_0$ & 
$u_2, \ldots, u_k, v_1, v_2, \ldots, v_k, w_1, w_2, \ldots, w_k$
& $u_0$ \\
\hline
$u_2$
& $u_3, v_2$
& $u_1$ & 
$u_3, \ldots, u_k, v_2, \ldots, v_k, w_2, \ldots, w_k$
& $u_0$ \\
\hline
 $\vdots$ & $\vdots$ & $\vdots$ &  $\vdots$ &  $\vdots$ \\
\hline
$u_k$
& $v_k$ 
& $u_{k-1}$ & 
$v_k, w_k$
& $u_0, u_1, \ldots, u_{k-1}$ \\
\hline
$v_1$
& $w_1$ 
& $u_1$ & 
$w_1$
& $u_0, u_1$ \\
\hline
$v_2$
& $w_2$
& $u_2$ & 
$w_2$
& $u_0, u_1, u_2$ \\
\hline
 $\vdots$ & $\vdots$ & $\vdots$ &  $\vdots$ &  $\vdots$ \\
\hline
$v_k$
& $w_k$
& $u_k$ & 
$w_k$
& $u_0, u_1, \ldots, u_k$ \\
\hline
$w_1$
& none
& $v_1$ & 
none
& $u_0, u_1, v_1$ \\
\hline
$w_2$
& none &
$v_2$ & 
none
& $u_0, u_1, u_2, v_2$ \\
\hline
$w_k$
& none
& $v_k$ & 
none
& $u_0, u_1, \ldots, u_k, v_k$
\end{tabular}
}
\end{center}
}
\caption{A tabular description of the rooted tree $\t$ of
  Figure \ref{fig.graph-samples}.  \label{tab.graph-samples}}
\end{table}
***************}



\section{Hypergraphs: an advanced topic}

When studying certain computing-related topics, one will
need a generalization of graphs called {\em hypergraphs}.  A
hypergraph has nodes that are analogous to the nodes of a graph, but
in place of edges a hypergraph has {\em hyperedges}.  Each hyperedge
is a set of nodes whose size is not restricted to $2$; thus,
hypergraphs represent internode relationships that are not ``binary''.
A simple example can be framed by describing {\em bus-connected}
systems, such as occur in certain genres of digital circuits and
certain message-passing systems.  The underlying idea is that nodes
that coreside in a hyperedge represent agents that ``hear'' all
messages simultaneously, or equi-potential points in a network.
Because of their inherent complexity, hypergraphs as graph-theoretic
objects are usually relegated to advanced courses, but specific
concrete instantiations, exemplified by bus-oriented communication and
digital circuits should be accessible even to early students.


